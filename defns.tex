% how to use glossary items
%
% Example:
% \newglossaryentry{Linux} {
%     name=Linux,
%     description={is a generic term referring to the family of Unix-like
%     computer operating systems that use the Linux kernel},
%     plural=Linuces
%     }
%
% \gls{<label>}   <-- substitutes "name"
% \Gls{<label>}   <-- substitutes "name" with first letter capitalized
% \glspl{<label>}   <-- substitutes "name" with an "s" on it or the "Plural" version
% \Glspl{<label>}   <-- substitutes "name" with an "s" on it or the "Plural" version and with first letter capitalized
% \glsdesc{<label>}   <-- substitutes the "description"
%

% how to use acronyms
%
% Example:
% \newacronym[longplural={Frames per Second}]{fpsLabel}{FPS}{Frame per Second}
%
% \acrshort{<label>}   <-- substitutes the abbreviation
% \acrlong{<label>}   <-- substitutes the expansion
% \acrfull{<label>}   <-- substitutes the expansion (abbreviation)
%

\newglossaryentry{seaker}{
    name={SEAKER},
    description={A digital forensics triage device built using a raspberry pi.
    It is an acronym for Storage Evaluator And Knowledge Extraction Reader}
    }

\newglossaryentry{rpi}{
    name={Raspberry Pi},
    description={A small, affordable computer created with the intention
    of providing low-cost computing power to the masses}
    }
