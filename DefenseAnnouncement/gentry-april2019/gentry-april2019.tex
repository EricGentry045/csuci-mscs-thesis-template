\documentclass{article}

\usepackage{graphicx}
\usepackage{amsmath}
\usepackage{float}
\usepackage{listings}
\usepackage{xcolor}
\usepackage[margin=0.7in]{geometry}
\usepackage{changepage}

\colorlet{grey}{gray!120}

\pagenumbering{gobble}

\begin{document}

\begin{center}

\includegraphics[width=0.3\textwidth]{CI_Logo.png}\\

\hfill\break

\LARGE
\textbf{\color{grey}Computer Science Master Thesis Presentation}\\

\hfill\break
\hfill\break

\Large
{\bf SEAKER: A Mobile Digital Forensic Triage Device}\\

\vspace{5mm}

\large
{\bf Eric Gentry}\\

\vspace{5mm}

\large
\textit{ Examination Committee:\\
{\bf Dr. Michael Soltys} (Advisor), {\bf Mr. Adam Wittkins}, {\bf
Dr. Jason Isaacs}}\\

\hfill\break

\end{center}

\begin{adjustwidth}{1in}{1in}
\textit{\bf Abstract:}\\

\vspace{3mm}

\normalsize
\noindent As our world of digital devices continues to
expand, the potential for digital evidence available to
law enforcement during case investigation is ever increasing.
The growing amount of digital evidence, along with the
considerable lack of Digital Forensic Investigators is
causing a backlog to form at many of the digital forensics
labs around the world. This backlog leads to delays in
evidence analysis and reporting, causing investigators and
prosecutors to postpone or even drop on-going cases.

The SEAKER device is a digital forensic triage tool that is
designed to be simple, portable, inexpensive, robust, and
easy to use. Utilizing a Raspberry Pi,
SEAKER is a novel approach to
providing immediate feedback to investigators. It is also
intended to help stem the backlog problem in digital
forensics labs worldwide.
\end{adjustwidth}

\hfill\break

\begin{center}

\LARGE
{\bf 2:00 pm, Wednesday, April 3\textsuperscript{rd}, 2019\\
Bell Tower 1642}\\

\vspace{15mm}

\large
{\bf All students and faculty are invited}\\

\hfill\break

\small
{\it An Academic Affairs Event}

\end{center}

\end{document}
